\subsection*{1.}
In tatate $x_j$ we can transition into state $x_i$, which will stop our process, or we can transition into some other state $x_k$, where our expected stopping time will increase by $1$, and the remainder is just the moment genersting function starting from $x_k$ to $x_i$.
\begin{gather*}
    F_j (s,x_i)=\cE{s^{T_{x_i}}}{X_0=x_j} = s^1 \cpr{X_1 = x_i}{X_0 = x_j} + \sum_{X_2 \neq x_i} \cE{s^{T_{x_i}+1}}{X_1=x_k} \cpr{X_1=x_k}{X_0=x_i} = \\
    s\cdot  p_{ji} + \sum_{i \neq k} s \cdot \cE{s^{T_{x_i}}}{X_1=x_k} p_{jk} = s \cdot p_{ji} +s  \sum_{i \neq k} F_k(s,x_i) p_{jk}
\end{gather*}
We can calculate the MGFs as follows
\begin{gather*}
    F_1(s,x_2) = 0.9s + 0.1 s F_1(s,x_2) \Longrightarrow  F_1(s,x_2) = \frac{0.9s}{1-0.1s} \\
    F_2(s,x_1) = 0.4s + 0.6 s F_2(s,x_2) \Longrightarrow F_2(s,x_1) = \frac{0.4s}{1-0.6s} \\
    F_3(s,x_1) = 0.3 s + 0.3 sF_2(s,x_1) + 0.4 sF_3(s,x_1) \Longrightarrow F_3(s,x_1)= \frac{1+\frac{0.4s}{1-0.6s}}{1-0.4s}0.3s\\
    F_3(s,x_2) = 0.3 s + 0.3s F_1(s,x_2) + 0.4 sF_3(s,x_2) \Longrightarrow F_3(s,x_2)= \frac{1+\frac{0.9s}{1-0.1s}}{1-0.4s}0.3s 
\end{gather*}
The calculations of $F_1(s,x_3)$ and $F_2(s,x_3)$ were omitted because starting from $x_1$ or $x_2$ $T_{x_3}$ will be infinity. W can also agree on $F_i(s,x_i)=1$.
With the MFGs, we can calculate the conditional expectations 
\begin{gather*}
    \cE{s^{T_{x_i}}}{X_0=x_j}=\sum_n s^{T_{x_i}}\ind{x_i=n}\pr{T_{x_i}=n} = \sum_n s^n \pr{T_{x_i}=n} \\
    \frac{\partial}{\partial s}F_j(s,x_i)\dummy{s=1} = \sum_n n s^{n-1} \pr{T_{x_i}=n}\dummy{s=1} = \sum_n n \pr{T_{x_i}=n}=\cE{T_{x_i}}{X_0=x_j} \\
    \frac{\partial^2}{\partial s^2}F_j(s,x_i)\dummy{s=1} = \sum_n n(n-1) s^{n-2} \pr{T_{x_i}=n}\dummy{s=1} = \sum_n n^2 \pr{T_{x_i}=n} - \sum_n n \pr{T_{x_i}=n} = \\ \cE{T^2_{x_i}}{X_0=x_j}- \cE{T_{x_i}=n}{X_0=x_j}
\end{gather*}
We can calculate the derivatives and plug in the results
\begin{gather*}
    \cE{T_{1}}{X_0=1} = 0 \\
    \cE{T_{1}}{X_0=2} =  \frac{10}{(3s-5)^2}\dummy{s=1}=\frac{10}{4}\\
    \cE{T_{1}}{X_0=3}  = -\frac{30(11s^2-40s-25)}{(s-10)^2(2s-5)^2}\dummy{s=1}=\frac{20}{9}\\
    \cE{T_{2}}{X_0=1} = \frac{90}{(s-10)^2}\dummy{s=1} =  \frac{90}{81}=\frac{10}{9}\\
    \cE{T_{2}}{X_0=2}  = 0\\
    \cE{T_{2}}{X_0=3}  = -\frac{15(s^2+10s-25)}{2(2s-5)^2(3s-5)^2}\dummy{s=1}= \frac{35}{12}\\
    \cE{T_{3}}{X_0=1} = \infty\\
    \cE{T_{3}}{X_0=2} = \infty\\
    \cE{T_{3}}{X_0=3} = 0 \\
\end{gather*}
Similarly we have
\begin{gather*}
    \cE{T_{1}^2}{X_0=1} = 0\\
    \cE{T_{1}^2}{X_0=2}  = -\frac{60}{(3s-5)^3}\dummy{s=1}-\frac{10}{4}\\
    \cE{T_{1}^2}{X_0=3} = -\frac{30(30s^3+45s^2-225s+250)}{(2s-5)^3(3s-5)^3}\dummy{s=1} - \frac{20}{9}\\
    \cE{T_{2}^2}{X_0=1}  = -\frac{180}{(s-10)^3}\dummy{s=1}-\frac{10}{9}\\
    \cE{T_{2}^2}{X_0=2}  = 0\\ 
    \cE{T_{2}^2}{X_0=3} =-\frac{60(22s^3-120s^2-150s+1625)}{(s-10)^3(2s-5)^3}\dummy{s=1}- \frac{35}{12}\\
    \cE{T_{3}^2}{X_0=1} = \infty\\
    \cE{T_{3}^2}{X_0=2} = \infty\\
    \cE{T_{3}^2}{X_0=3}  = 0\\
\end{gather*}


\subsection*{2.}
\subsection*{3.}
In this exercise, we want to use the Optional Sampling Theorem. The stopping time here is $\tau = \{t : M_t = 0 \text{ or } M_t =50\}$
Here the stopping time is almost surely bonded, because we are talking about a simple random walk, which visits all levels surely in finite time. Also, if we denote the amount of money at time $t$ with $M_t$, then $|M_{t+1}- M_t| = 1 $, so bounded. As a result of the theorem $\E{M_\tau} = \E{M_0} = 7$. As for the other question, we can use what we derived before. The following holds, because the stopping the is almost surely finite. $\E{M_\tau} = \pr{M_\tau = 0} \cdot 0+ \pr{M_\tau = 50} \cdot 50 = 7 $, so the probability is $\frac{7}{50}$
\subsection*{4.}

\textcolor{blue}{I think we don't have to use the OST, it is enough that a martingal has a constant expected value, $S(t)$ is a martingal so its expected value will be equal to the expected value of $S(0)$ which is $47$} \\
Let's consider $N$ a constant stopping time and prove that the Optional Sampling Theorem holds.\\
First of all, $N$ is finite. Moreover, the price of the contract is bounded since $\lim_{n \to \infty} S(n) = S(N)$ a.s. as $N$ is finite and $S(n)$ is a martingale, so so is the stopped process (see the theorem on slide 131/697).\\
The above-mentioned facts satisfy the (ii) conditions of the Optional Sampling Theorem (see slide 133/697).\\
Then,
\begin{gather*}
    E[S(N)] = S(0) = 47
\end{gather*}
On the other hand,
\begin{gather*}
    E[S(N)] = 100P(S(N)=100)+0P(S(N)=0) = 100P(S(N)=100)
\end{gather*}
So $100P(S(N)=100)=47 \implies \mathbf{P(S(N)=100) = \frac{47}{100}}$

\subsection*{5.}
In this exercise our stopping time is $T$ and we can also prove that the (ii) conditions of the Optional Sampling Theorem hold, as in the previous exercise.\\
\begin{enumerate}
    \item $T$ is finite.\\
    In the "worst-case scenario", if the price of the contract does not go above 55 or below 15 $\forall t \in (0,N)$ at time $T=N$ the contract will be worth either 100 or 0, so Pedro will sell it.\\
    Therefore, T is finite, as $T \in (0,N]$.
\end{enumerate}